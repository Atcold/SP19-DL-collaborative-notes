\chapter{The Manifold Hypothesis}\label{chp: manifold_hypothesis}
% Authors: Hongyu (Florence) Lu, Michael Gold, Erica Dominic.
% Lecture date: 1.28.19

\section{Facial Expressions Thought Experiment}
% Authors: Hongyu (Florence) Lu, Michael Gold, Erica Dominic.
% Lecture date: 1.28.19

Say we have infinitely many pictures of a person making all possible facial expressions.
Each image is $2000 \times 1000$ pixels and has $3$ color channels, so each image is $6,000,000$-dimensional vector.

The set of all images is a small subset of $6,000,000$-dimensional space.
What does that subset look like?
What is the dimensions of the surface?
On a patch of that surface, how many dimensions can you move and still stay on that surface?

That surface is a manifold (roughly speaking, a continuous surface).
Moreover, it is limited by the number of degrees of freedom in the human face, which is bounded above by the number of muscle groups a person can control in his face.
Ergo that subset of $\mathbb{R}^{6000000}$ is relatively low-dimensional.

This thought experiment illustrates the manifold hypothesis, which postulates that natural data in high dimensional space generally has a low dimensional structure.
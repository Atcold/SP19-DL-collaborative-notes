\chapter{Variational Auto-Encoder}
% Authors: Yu Cao, Evgenii Nikitin, Aishwarya Budhkar (editor), 4/16/2019

In this chapter, we present the a module to achieve variational auto-encoder(VAE). 
\section{Setup}
\begin{minted}{python}
import torch
import torchvision
from torch import nn
from torch.utils.data import DataLoader
from torchvision import transforms
from torchvision.datasets import MNIST
from matplotlib import pyplot as plt

# Displaying routine
def display_images(in_, out, n=1, label=None, count=False):
    for N in range(n):
        if in_ is not None:
            in_pic = in_.data.cpu().view(-1, 28, 28)
            plt.figure(figsize=(18, 4))
            plt.suptitle(label + ' – real test data / reconstructions', color='w', fontsize=16)
            for i in range(4):
                plt.subplot(1,4,i+1)
                plt.imshow(in_pic[i+4*N])
                plt.axis('off')
        out_pic = out.data.cpu().view(-1, 28, 28)
        plt.figure(figsize=(18, 6))
        for i in range(4):
            plt.subplot(1,4,i+1)
            plt.imshow(out_pic[i+4*N])
            plt.axis('off')
            if count: plt.title(str(4 * N + i), color='w')
            
# Set random seeds

torch.manual_seed(1)
torch.cuda.manual_seed(1)

# Define data loading step

batch_size = 256

kwargs = {'num_workers': 1, 'pin_memory': True}
train_loader = torch.utils.data.DataLoader(
    MNIST('./data', train=True, download=True,
                   transform=transforms.ToTensor()),
    batch_size=batch_size, shuffle=True, **kwargs)
test_loader = torch.utils.data.DataLoader(
    MNIST('./data', train=False, transform=transforms.ToTensor()),
    batch_size=batch_size, shuffle=True, **kwargs)
    
device = torch.device("cuda:0" if torch.cuda.is_available() else "cpu")
\end{minted}

\section{VAE model}

\begin{minted}{python}
d = 20

class VAE(nn.Module):
    def __init__(self):
        super().__init__()

        self.encoder = nn.Sequential(
            nn.Linear(784, d ** 2),
            nn.ReLU(),
            nn.Linear(d ** 2, d * 2)
        )

        self.decoder = nn.Sequential(
            nn.Linear(d, d ** 2),
            nn.ReLU(),
            nn.Linear(d ** 2, 784),
            nn.Sigmoid(),
        )

    def reparameterise(self, mu, logvar):
        if self.training:
            std = logvar.mul(0.5).exp_()
            eps = std.data.new(std.size()).normal_()
            return eps.mul(std).add_(mu)
        else:
            return mu

    def forward(self, x):
        mu_logvar = self.encoder(x.view(-1, 784)).view(-1, 2, d)
        mu = mu_logvar[:, 0, :]
        logvar = mu_logvar[:, 1, :]
        z = self.reparameterise(mu, logvar)
        return self.decoder(z), mu, logvar

model = VAE().to(device)
\end{minted}

Here we define our VAE. We define an encoder mapping Linear $\rightarrow$ RELU $\rightarrow$ Linear.
The output of the encoder is of 2d dimension because we need mean and variance for further sampling. The input in the decoder is of d dimension because for input we used the two parameters returned by encoder to generate, mean and variance and sampled it. The 784 corresponds to the MNIST data dimension (28 * 28). We use log-variance so that we can use the whole range of data instead of only positive numbers using variance. We get our latent variable ($z$ in our case) by multiplying  deviation(generated by encoder) to the epsilon(sampled from a normal distribution) and add our mu(generated by encoder). We do not backpropagate through sampling(we only care about variance and mean produced by our encoder, not episilon).

\section{Training}
\begin{minted}{python}
learning_rate = 1e-3

optimizer = torch.optim.Adam(
    model.parameters(),
    lr=learning_rate,
)
# Reconstruction + KL divergence losses summed over all elements and batch

def loss_function(x_hat, x, mu, logvar):
    BCE = nn.functional.binary_cross_entropy(
        x_hat, x.view(-1, 784), reduction='sum'
    )
    KLD = -0.5 * torch.sum(1 + logvar - mu.pow(2) - logvar.exp())

    return BCE + KLD
    
# Training and testing the VAE

epochs = 11
for epoch in range(1, epochs + 1):
    # Training
    model.train()
    train_loss = 0
    for x, _ in train_loader:
        x = x.to(device)
        # ===================forward=====================
        x_hat, mu, logvar = model(x)
        loss = loss_function(x_hat, x, mu, logvar)
        train_loss += loss.item()
        # ===================backward====================
        optimizer.zero_grad()
        loss.backward()
        optimizer.step()
    # ===================log========================
    print(f'====> Epoch: {epoch} Average loss: {train_loss / len(train_loader.dataset):.4f}')
    
    # Testing
    
    with torch.no_grad():
        model.eval()
        test_loss = 0
        for x, _ in test_loader:
            x = x.to(device)
            # ===================forward=====================
            x_hat, mu, logvar = model(x)
            test_loss += loss_function(x_hat, x, mu, logvar).item()
    # ===================log========================
    test_loss /= len(test_loader.dataset)
    print(f'====> Test set loss: {test_loss:.4f}')
    display_images(x, x_hat, 1, f'Epoch {epoch}')
\end{minted}

\section{Result}
\subsection{Random sample + Decoder}
\begin{minted}{python}
# Generating a few samples

N = 16
sample = torch.randn((N, d)).to(device)
sample = model.decoder(sample)
display_images(None, sample, N // 4, count=True)
\end{minted}


\includegraphics[scale=0.3]{"labs/11/images/random+decoder".png}


Above is an example of generating something given samples drawn in the latent space with normal distribution instead of using encoder. In the above case 8 was generated starting from random sample in normal distribution. The third picture already looks like an 8.

\subsection{Smooth interpolation} 
\begin{minted}{python}
# Perform an interpolation between input A and B, in N steps

N = 16
code = torch.Tensor(N, 20).to(device)
for i in range(N):
    code[i] = i / (N - 1) * mu[B].data + (1 - i / (N - 1) ) * mu[A].data
sample = model.decoder(code)
display_images(None, sample, N // 4, count=True)
\end{minted}

\includegraphics[width=8cm]{"labs/11/images/interpolation".png}

Above we present a smooth interpolation from an 8 to a 3, in 16 steps.

Linear interpolation in the latent space is warped. 